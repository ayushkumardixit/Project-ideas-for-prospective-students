\documentclass[a4paper,10pt]{article}
\usepackage{fullpage}
\usepackage{float}
\usepackage[english]{babel}
\usepackage{graphicx,subfig,wrapfig}
\usepackage{amsmath,amsfonts,amsthm,amssymb}
\usepackage{fancyhdr,fancybox,color}
\usepackage{enumerate}
\usepackage[amssymb]{SIunits}
\definecolor{MyBlue}{rgb}{0,0.3,0.6}
\usepackage[colorlinks=true,
linkcolor=MyBlue,
plainpages=false,
citecolor=MyBlue,
urlcolor=MyBlue]{hyperref}
\usepackage[all]{hypcap}
\usepackage[url=false,
backend=bibtex,
style=authoryear-comp,
doi=true,
isbn=true,
backref=false,
dashed=false,
maxcitenames=2,
maxbibnames=99,
natbib=true]{biblatex}
\DeclareNameAlias{author}{last-first}
\renewbibmacro{in:}{}
\addbibresource{refrence.bib}
\nonfrenchspacing
\begin{document}
\noindent Chair: Physics of Fluids group
\begin{center}
 \begin{LARGE}
  Bursting Bubbles: from champagne to complex liquids
 \end{LARGE}
\end{center}
\section*{Description}
Interaction of gas bubbles with the free liquid-gas interface is ubiquitous. For example, transporting aromatics from champagne and pathogens from contaminated water. Furthermore, the process is also responsible for the sea spray formation as a consequence of ejecting myriads of droplets.\\
First, the air bubble, generated in the liquid bulk, being lighter than the surrounding medium, rises because of buoyancy and reaches the liquid-air interface. It stays there as the thin film between the bubble and the free surface drains and ruptures to form film droplets \citep{lhuissier2012bursting}. The rupture results in the formation of a hole in the liquid meniscus. The unstable open cavity collapses leading to the interaction of the capillary waves and forms of an upward jet.
Figure~\ref{Figure::Typical} illustrates a typical temporal sequence of the process. The phenomenon has been widely studied by \cite{duchemin2002jet, walls2015jet, deike2018dynamics, gordillo2019capillary}.
\cite{deike2018dynamics} has provided quantitative cross-validation of the numerical and experimental studies (Figure~\ref{Figure::Waves}). Nonetheless, the literature lacks a comprehensive study on the influence of liquid pool properties on the bubble bursting process. Notably, the influence of rheological properties on the scaling laws can provide a close-up to the understanding of the phenomenon. This latter is highly related to several geophysical phenomena such as mudpots where viscoplasticity matters (see \citet{sanjay_lohse_jalaal_2021} and \href{https://www.youtube.com/watch?v=a9hUsVq9q7U}{https://www.youtube.com/watch?v=a9hUsVq9q7U}), and spread of microbes where viscoelasticity is essential \citet{walls2017quantifying, bourouiba2021fluid}.
\begin{figure}[H]
\begin{center}
 \includegraphics[width=\textwidth]{temporal.png}
 \caption{Time resolution of the process. The color shows the magnitude of non-dimensionalized vorticity, $\Gamma$ (maximum $\Gamma$ = 150 with red and minimum $\Gamma$ = -150 with blue).}
 \label{Figure::Typical}
\end{center}
\end{figure}
\begin{figure}[H]
\begin{center}
 \includegraphics[width=0.85\textwidth]{interface.eps}
 \caption{Collapse of the bubble cavity and interaction of capillary waves}
 \label{Figure::Waves}
\end{center}
\end{figure}
\section*{What you will do and what you will learn?}
% The project will focus on the following:
\begin{enumerate}
\item You will learn about the fundamental fluid mechanics of two-phase flows.
\item You will learn the science of rheology.
\item You will learn the freeware code \href{http://basilisk.fr}{Basilisk C} to simulate fluid dynamics problems.
%\item Validation of the numerical model using the scaling laws available in the literature for the collapse of air bubble cavity in Newtonian fluid pool \citep{duchemin2002jet}.
%\item Implementation and validation of the generalized Newtonian fluid viscosity model to the free-code \href{http://basilisk.fr}{Basilisk C }.
%\item Studying the effects of a Viscoplastic liquid medium on the bursting of bubbles at the interface.
%\item Understanding the dependence of the liquid's viscoplasticity on the proposed scaling laws.
\end{enumerate}

If you have any questions, fell free to contact \href{mailto:vatsalsanjay@gmail.com}{Vatsal} or \href{mailto:a.k.dixit@utwente.nl}{Ayush} (details below).
\begin{center}
\begin{tabular}{|l|l|l|}
\hline \textbf{Supervision} & \textbf{E-mail} & \textbf{Office} \\
\hline Dr. Vatsal Sanjay & \href{mailto:vatsalsanjay@gmail.com}{vatsalsanjay@gmail.com} & Meander 246B \\
\hline Ayush Dixit & \href{mailto:a.k.dixit@utwente.nl}{a.k.dixit@utwente.nl} & Meander 114B \\
\hline Dr. Alexandros Oratis   & \href{mailto:a.t.oratis@utwente.nl}{a.t.oratis@utwente.nl}& Meander 250 \\
\hline Assis. Prof. Dr. Maziyar (Mazi) Jalaal   & \href{mailto:m.jalaal@uva.nl}{m.jalaal@uva.nl}& University of Amsterdam \\
\hline Prof. Dr. D. Lohse & \href{mailto:d.lohse@utwente.nl}{d.lohse@utwente.nl} & Meander 261  \\
\hline
\end{tabular}
\end{center}
\printbibliography
\end{document}
